\documentclass{article}

\usepackage{hyperref}
\usepackage{fancyhdr}
\usepackage{graphicx}
\usepackage{caption}
\usepackage{subcaption}
\usepackage{babel, blindtext}
%\lhead{\includegraphics[width=0.2\textwidth]{nyush-logo.pdf}}
\fancypagestyle{firstpage}{%
  \lhead{Course Project}
  \rhead{
  Operating Systems}
}

%%%% PROJECT TITLE
\title{ \LARGE Voice Assisstant Powered by AI
        }

%%%% NAMES OF ALL THE STUDENTS INVOLVED (first-name last-name)
\author{\href{mailto:arora.8@iitj.ac.in}{Piyush Arora} \vspace*{1} \href{mailto:arora.8@iitj.ac.in}{Kartik Choudhary} \vspace*{1} \href{mailto:arora.8@iitj.ac.in}{Kartik Chippa}}

\date{\vspace{-5ex}} %NO DATE


\begin{document}
\maketitle
\thispagestyle{firstpage}



\begin{center}
    \large IIT Jodhpur, India
\end{center}

\section*{Introduction}
We are all well aware of Cortana, Siri, Google Assistant, and many other virtual assistants designed to aid users with Windows, Android, and iOS platforms. Artificial Intelligence personal assistants have become plentiful over the last few years. 
But to our surprise, there is no complete voice assistant available on Linux platforms with an appreciable number of users. 
Moreover, why are we still learning terminal commands? Is there a way to use terminal commands without cramming them? Is there a way to teach beginners to operate with terminal statements? New users generally find it hard to learn terminal commands. Can we use advancements in AI to tackle this problem and provide a user-friendly solution? 
We are daring to put a step forward in this direction.

\section*{Proposed Solution}
(mention the working of bert model here: yahan likh de working or pictures laga dena of possible)
Training BERT (Bidirectional Encoder Representations from Transformers) which is an open-source machine learning framework for natural language processing (NLP) on our own dataset using Google's Dialogflow
Fetching output from cloud-hosted BERT model using python scripts
The response is stored in a JSON file and then processed by our scripts.
The results are passed through a module that converts the text into a voice that is heard by the user.
User Interface build on Tkinter 
(paste the picture of UI)

This proposed pipeline helps us to interact with the system using voice commands and establishes a connection between AI and terminal

\section*{Outcomes}
The main outcomes The fully functional voice assistant that understands natural language
A step forward in powering terminals with AI
Tried to create our own datasets using Dialogflow to train BERT
Helping beginners with learning to use terminal commands with a user-friendly GUI
Adding voice and elements of NLP to enhance the functionality of terminal commands

\section*{Lessons Learnt}
AI to understand Language: We learned to use cloud-based language models in order to take terminal to next level

A friendly user interface: We learned to use Tkinter, a popular python library to provide the users with a friendly interface

I/O using Voice: We learned to use voice as a source for input and output to a Linux-based system

OS system calls: We learned to use system calls in order to simulate terminal commands

\bibliographystyle{IEEEtran}
\bibliography{references}



\end{document}
